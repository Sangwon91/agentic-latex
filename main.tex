\documentclass[12pt,a4paper]{article}

% 필수 패키지들
\usepackage[utf8]{inputenc}
\usepackage[T1]{fontenc}
\usepackage[korean,english]{babel}
\usepackage{kotex}
\usepackage{amsmath,amsfonts,amssymb}
\usepackage{graphicx}
\usepackage[colorlinks=true,linkcolor=blue,citecolor=red,urlcolor=blue]{hyperref}
\usepackage{cite}
\usepackage{url}
\usepackage{booktabs}
\usepackage{array}
\usepackage{subcaption}
\usepackage{algorithm}
\usepackage{algorithmic}
\usepackage{listings}
\usepackage{xcolor}

% 페이지 설정
\usepackage[margin=2.5cm]{geometry}
\usepackage{setspace}
\onehalfspacing

% 코드 스타일 설정
\lstset{
    backgroundcolor=\color{gray!10},
    basicstyle=\ttfamily\small,
    breaklines=true,
    numbers=left,
    numberstyle=\tiny,
    frame=single,
    showstringspaces=false
}

% 제목과 저자 정보
\title{안녕 논문임}
\author{
    홍길동\thanks{교신저자: gildong@example.com} \\
    \textit{컴퓨터과학과} \\
    \textit{예시대학교} \\
    \textit{서울, 대한민국}
    \and
    김철수 \\
    \textit{인공지능학과} \\
    \textit{예시대학교} \\
    \textit{서울, 대한민국}
}
\date{\today}

\begin{document}

\maketitle

\begin{abstract}
본 논문의 초록을 여기에 작성하세요. 연구의 목적, 방법, 주요 결과, 그리고 결론을 간략하게 요약해서 설명합니다. 초록은 보통 150-250단어 정도로 작성하며, 독자가 논문의 전체 내용을 빠르게 파악할 수 있도록 핵심 내용을 담아야 합니다.
\end{abstract}

\textbf{키워드:} 키워드1, 키워드2, 키워드3, 키워드4, 키워드5

\section{서론}
\label{sec:introduction}

연구의 배경과 동기를 설명하고, 해결하고자 하는 문제를 명확히 제시합니다. 

\subsection{연구 배경}
연구 분야의 현재 상황과 문제점을 설명합니다.

\subsection{연구 목적}
본 연구에서 달성하고자 하는 목표를 구체적으로 기술합니다.

\subsection{논문 구성}
논문의 각 장별 구성과 내용을 간략히 소개합니다. 2장에서는 관련 연구를, 3장에서는 제안하는 방법론을, 4장에서는 실험 결과를, 마지막으로 5장에서는 결론을 다룹니다.

\section{관련 연구}
\label{sec:related_work}

기존 연구들을 정리하고 분석합니다. 각 연구의 장단점을 객관적으로 평가하고, 본 연구와의 차별점을 명확히 제시합니다.

\subsection{기존 접근법들}
관련 연구 분야의 주요 접근법들을 분류하여 설명합니다.

\subsection{기존 연구의 한계점}
선행 연구들의 한계점과 개선이 필요한 부분을 분석합니다.

\section{제안 방법론}
\label{sec:methodology}

본 연구에서 제안하는 방법론을 상세히 기술합니다.

\subsection{전체 시스템 개요}
제안하는 시스템의 전체적인 구조와 동작 원리를 설명합니다.

\begin{figure}[htbp]
    \centering
    % \includegraphics[width=0.8\textwidth]{figures/system_overview.png}
    \caption{제안하는 시스템의 전체 구조도}
    \label{fig:system_overview}
\end{figure}

\subsection{핵심 알고리즘}
제안하는 방법론의 핵심 알고리즘을 기술합니다.

\begin{algorithm}
\caption{제안하는 알고리즘}
\label{alg:proposed}
\begin{algorithmic}[1]
\REQUIRE 입력 데이터 $X$
\ENSURE 결과 $Y$
\STATE 초기화
\FOR{각 데이터 포인트 $x_i \in X$}
    \STATE 처리 단계 1
    \STATE 처리 단계 2
\ENDFOR
\RETURN $Y$
\end{algorithmic}
\end{algorithm}

\subsection{이론적 분석}
제안하는 방법의 시간 복잡도, 공간 복잡도 등을 분석합니다.

\section{실험 및 결과}
\label{sec:experiments}

실험 설정, 데이터셋, 평가 지표, 그리고 실험 결과를 상세히 기술합니다.

\subsection{실험 환경}
실험에 사용된 하드웨어, 소프트웨어 환경을 명시합니다.

\subsection{데이터셋}
실험에 사용된 데이터셋의 특성과 전처리 과정을 설명합니다.

\subsection{평가 지표}
성능 평가에 사용된 지표들을 정의하고 선택 이유를 설명합니다.

\subsection{실험 결과}
\begin{table}[htbp]
\centering
\caption{실험 결과 비교}
\label{tab:results}
\begin{tabular}{@{}lcccc@{}}
\toprule
방법론 & 정확도 & 정밀도 & 재현율 & F1-점수 \\
\midrule
기존 방법 1 & 85.2\% & 82.1\% & 88.3\% & 85.1\% \\
기존 방법 2 & 87.5\% & 85.2\% & 89.1\% & 87.1\% \\
\textbf{제안 방법} & \textbf{92.3\%} & \textbf{90.1\%} & \textbf{94.2\%} & \textbf{92.1\%} \\
\bottomrule
\end{tabular}
\end{table}

\subsection{결과 분석}
실험 결과를 분석하고 제안하는 방법의 우수성을 입증합니다.

\section{논의}
\label{sec:discussion}

실험 결과에 대한 심층적인 분석과 해석을 제공합니다.

\subsection{성능 향상의 원인}
제안하는 방법이 우수한 성능을 보이는 이유를 분석합니다.

\subsection{한계점}
연구의 한계점과 향후 개선 방향을 논의합니다.

\section{결론}
\label{sec:conclusion}

연구의 주요 기여점을 요약하고, 향후 연구 방향을 제시합니다.

\subsection{주요 기여점}
본 연구의 주요 성과와 기여점을 정리합니다.

\subsection{향후 연구}
현재 연구의 한계를 극복하고 발전시킬 수 있는 향후 연구 방향을 제시합니다.

% 감사의 글
\section*{감사의 글}
연구 지원에 대한 감사 인사를 기술합니다.

% 참고문헌
\bibliographystyle{ieeetr}
\bibliography{references}

\end{document} 